%%%%%%%%%%%%%%%%%%%%%%%%%%%%%%%%%%%%%%%%%%%%%%%%%%%%%%%%%%%%%%%%%%%%%%
% How to use writeLaTeX: 
%
% You edit the source code here on the left, and the preview on the
% right shows you the result within a few seconds.
%
% Bookmark this page and share the URL with your co-authors. They can
% edit at the same time!
%
% You can upload figures, bibliographies, custom classes and
% styles using the files menu.
%
%%%%%%%%%%%%%%%%%%%%%%%%%%%%%%%%%%%%%%%%%%%%%%%%%%%%%%%%%%%%%%%%%%%%%%

\documentclass[12pt]{article}

\usepackage{sbc-template}

\usepackage{graphicx,url}

%\usepackage[brazil]{babel}   
\usepackage[utf8]{inputenc}  

     
\sloppy

\title{Relatório: Trabalho Prático 2\\ Soluções para problemas difíceis}

\author{ Marcos Paulo F. de Souza\inst{1} }


\address{
  Departamento de Ciência da Computação\\
  Universidade Federal de Minas Gerais (UFMG) -- Belo Horizonte, MG -- Brazil
  \email{marcos.ferreira@dcc.ufmg.br}
}

\begin{document} 

\maketitle

\begin{abstract}
  This meta-paper describes the style to be used in articles and short papers
  for SBC conferences. For papers in English, you should add just an abstract
  while for the papers in Portuguese, we also ask for an abstract in
  Portuguese (``resumo''). In both cases, abstracts should not have more than
  10 lines and must be in the first page of the paper.
\end{abstract}
     
\begin{resumo} 
  Este meta-artigo descreve o estilo a ser usado na confecção de artigos e
  resumos de artigos para publicação nos anais das conferências organizadas
  pela SBC. É solicitada a escrita de resumo e abstract apenas para os artigos
  escritos em português. Artigos em inglês deverão apresentar apenas abstract.
  Nos dois casos, o autor deve tomar cuidado para que o resumo (e o abstract)
  não ultrapassem 10 linhas cada, sendo que ambos devem estar na primeira
  página do artigo.
\end{resumo}


\section{Introdução}

Lorem Ipson

\section{Conceitos Básicos}

O problema do caixeiro viajante foi apresentado por Anany Levitin em seu livro chamado "Introduction to The Design and Analysis of Algorithms". A sucinta definição trazida foi a de encontrar encontrar a menor rota possível passando por N cidades, apenas uma vez por cidade, e depois retornando para a cidade de origem. Esse problema é classificado como  NP-Difícil, portanto ele pode ser resolvido em tempo polinomial por uma máquina de Turing não deterministica. Isso significa que os computadores atuais não são capazes que encontrar a uma solução para esse problema em tempo hábil. Por isso, são precisos algoritmos mais complexos para encontrar uma solução mais rápida. Para este trabalho serão implementados três algoritmos que resolvem esse problema. O primeiro algoritmo implementado usa o Branch and Bound, o segundo é o algoritmo aproximativo Twice-around-the-tree e o terceiro é o algoritmo aproximativo Christofides. Toda a definição das implementações foram retiradas do livro "Introduction to The Design and Analysis of Algorithms".

\section{Implementação}

A implementação dos três algoritmos foi feita em um mesmo arquivos na linguagem de programação chamada python. Foram usadas as bibliotecas math e networkx para a implementação dos algoritmos. As bibliotecas os e timeout\_decorator foram usadas durante os teste para automatizar a execução dos teste e limitar o tempo de execução de cada algoritmo respectivamente. Já as bibliotecas <incluir> foram usadas para gerar as tabelas e os resultados para serem analisados.

Os três altoritmos terão as suas implemenações detalhadas a seguir:

\subsection{Branch and Bound}

Para a implementação desse algoritmo a estrutura de dados Graph disponibilizada pela biblioteca networkx foi utilizada, além de alguns métodos auxiliares que foram implementados a parte e estão contidos no mesmo arquivo dos demais algoritmos.

Para ser um algoritmo Branch and Bound é preciso ter o calculo de uma estimativa que seja realista e proxima ao valor otimo. O calculo da estimativa que foi implementado foi o mesmo descrito pro Levitin no livro "Introduction to The Design and Analysis of Algorithms". Em resumo, a estimativa é a soma das duas arestas de menor custo de todos os nós do grafo divida por dois. Quando uma aresta é introduzida na possivel solução ela é considerada como uma das menores arestas dos nós que ela liga. E isso garante que a estimativa de uma provavel solução parcial seja realista.

A arvoré de soluções possiveis é armazenada como uma fila de prioridade onde a prioridade é da solução que possui a menor estimativa. Ou seja, para avançar na arvoré de soluções possiveis é usada a busca em largura, escolhendo sempre a solução com menor estimativa e não a que esteja mais profunda na árvore.

\subsection{Twice-around-the-tree}

Para a implementação desse algoritmo a estrutura de dados Graph e os algoritmos minimum\_spanning\_tree e dfs\_preorder\_nodes que são disponibilizados pela biblioteca networkx foram utilizados.

A implementação desse algoritmo foi feita como é descrita no livro "Introduction to The Design and Analysis of Algorithms". São três passos a serem feitos, primeiro calcular a árvore geradora minima para o grafo da instancia do problema, segundo passo é definir a lista de vertices da arvores geradora minima a partir de visitação preorder dos vertices e terceiro e ultimo passo retirar do lista de vertices os vertices repetidos encontrando o ciclo hamiltoniano do grafo que é a solução para o problema.

\subsection{Christofides}

Para a implementação desse algoritmo a estrutura de dados Graph e os algoritmos minimum\_spanning\_tree, min\_weight\_matching e eulerian\_circuit que são disponibilizados pela biblioteca networkx foram utilizados, além de um método auxiliar que foi implementado a parte e está contido no mesmo arquivo dos demais algoritmos.

A implementação desse algoritmo foi feita como é descrita no livro "Introduction to The Design and Analysis of Algorithms". Essa implementação é semelhante a do algoritmo Twice-around-the-tree porém com algumas alterações. Primeiro é necessario encontrar a arvore geradora minima e depois gerar uma lista de nós da arvore geradora minima que possuem grau impar. Dentro do subgrafo induzido pelos nós de grau impar da arvore geradora minima é encontrado o matching perfeito de peso minimo. Com isso será criado um grafo auxiliar que adiciona na arvore geradona minima todas as arestas do matching perfeito de peso minimo. Nesse grafo auxiliar é calculado o circuito euleriano e com a lista de vertices do circuito euleriano basta retirar os vertices repetidos e encontraremos o circuito hamiltoniano que é a solução para o problema.

\section{Testes}

Os algoritmos implementados foram testados a partir das intancias contidas no site \url{http://comopt.ifi.uni-heidelberg.de/software/TSPLIB95/tsp/}. Dentre todas as instancias disponiveis apenas as euclidianas com 2 dimenões e instancias do caxeiro viajante foram usadas, totalizando 78 testes. Desses testes 6 que possuem mais do que 7000 nós não puderam ser executados, pois o computador utilizado para executar os testes trava e não termina os testes com essa quantia de nós.

\subsection{Resultados}

Lorem Ipson

\section{Conclusão}

Lorem Ipson


\section{General Information}

All full papers and posters (short papers) submitted to some SBC conference,
including any supporting documents, should be written in English or in
Portuguese. The format paper should be A4 with single column, 3.5 cm for upper
margin, 2.5 cm for bottom margin and 3.0 cm for lateral margins, without
headers or footers. The main font must be Times, 12 point nominal size, with 6
points of space before each paragraph. Page numbers must be suppressed.

Full papers must respect the page limits defined by the conference.
Conferences that publish just abstracts ask for \textbf{one}-page texts.

\section{First Page} \label{sec:firstpage}

The first page must display the paper title, the name and address of the
authors, the abstract in English and ``resumo'' in Portuguese (``resumos'' are
required only for papers written in Portuguese). The title must be centered
over the whole page, in 16 point boldface font and with 12 points of space
before itself. Author names must be centered in 12 point font, bold, all of
them disposed in the same line, separated by commas and with 12 points of
space after the title. Addresses must be centered in 12 point font, also with
12 points of space after the authors' names. E-mail addresses should be
written using font Courier New, 10 point nominal size, with 6 points of space
before and 6 points of space after.

The abstract and ``resumo'' (if is the case) must be in 12 point Times font,
indented 0.8cm on both sides. The word \textbf{Abstract} and \textbf{Resumo},
should be written in boldface and must precede the text.

\section{CD-ROMs and Printed Proceedings}

In some conferences, the papers are published on CD-ROM while only the
abstract is published in the printed Proceedings. In this case, authors are
invited to prepare two final versions of the paper. One, complete, to be
published on the CD and the other, containing only the first page, with
abstract and ``resumo'' (for papers in Portuguese).

\section{Sections and Paragraphs}

Section titles must be in boldface, 13pt, flush left. There should be an extra
12 pt of space before each title. Section numbering is optional. The first
paragraph of each section should not be indented, while the first lines of
subsequent paragraphs should be indented by 1.27 cm.

\subsection{Subsections}

The subsection titles must be in boldface, 12pt, flush left.

\section{Figures and Captions}\label{sec:figs}


Figure and table captions should be centered if less than one line
(Figure~\ref{fig:exampleFig1}), otherwise justified and indented by 0.8cm on
both margins, as shown in Figure~\ref{fig:exampleFig2}. The caption font must
be Helvetica, 10 point, boldface, with 6 points of space before and after each
caption.

\begin{figure}[ht]
\centering
\includegraphics[width=.5\textwidth]{fig1.jpg}
\caption{A typical figure}
\label{fig:exampleFig1}
\end{figure}

\begin{figure}[ht]
\centering
\includegraphics[width=.3\textwidth]{fig2.jpg}
\caption{This figure is an example of a figure caption taking more than one
  line and justified considering margins mentioned in Section~\ref{sec:figs}.}
\label{fig:exampleFig2}
\end{figure}

In tables, try to avoid the use of colored or shaded backgrounds, and avoid
thick, doubled, or unnecessary framing lines. When reporting empirical data,
do not use more decimal digits than warranted by their precision and
reproducibility. Table caption must be placed before the table (see Table 1)
and the font used must also be Helvetica, 10 point, boldface, with 6 points of
space before and after each caption.

\begin{table}[ht]
\centering
\caption{Variables to be considered on the evaluation of interaction
  techniques}
\label{tab:exTable1}
\includegraphics[width=.7\textwidth]{table.jpg}
\end{table}

\section{Images}

All images and illustrations should be in black-and-white, or gray tones,
excepting for the papers that will be electronically available (on CD-ROMs,
internet, etc.). The image resolution on paper should be about 600 dpi for
black-and-white images, and 150-300 dpi for grayscale images.  Do not include
images with excessive resolution, as they may take hours to print, without any
visible difference in the result. 

\section{References}

Bibliographic references must be unambiguous and uniform.  We recommend giving
the author names references in brackets, e.g. \cite{knuth:84},
\cite{boulic:91}, and \cite{smith:99}.

The references must be listed using 12 point font size, with 6 points of space
before each reference. The first line of each reference should not be
indented, while the subsequent should be indented by 0.5 cm.

\bibliographystyle{sbc}
\bibliography{sbc-template}

\end{document}
